\documentclass[11pt]{article}
\usepackage{scimisc-cv}
\usepackage{fontawesome}
\usepackage{hyperref}
% \def\ci#1{\textcircled{\resizebox{.5em}{!}{#1}}}

\newcommand{\mail}{ad945@cam.ac.uk}
\newcommand{\name}{Anish Das}
\newcommand{\firstName}{Anish}
\newcommand{\lastName}{Das}
\newcommand{\githubLink}{https://github.com/DasAnish}
\newcommand{\github}{DasAnish}
\newcommand{\linkedIn}{anish-das-ad945}
\newcommand{\linkedInLink}{https://www.linkedin.com/in/anish-das-ad945/}

\title{Anish Das CV}
\author{Anish Das}
\date{Sep 2020}

%% These are custom commands defined in scimisc-cv.sty
\cvname{Anish Das}
\cvpersonalinfo{
% Trinity Hall, Trinity Lane, CB \cvinfosep 
%\href{mailto:\mail}
{{\faEnvelope}  \mail} \cvinfosep
{{\faPhone} +44-7379444416} \cvinfosep
%\href{\githubLink}
{{\faGithub} \github} \cvinfosep
%\href{\linkedInLink}
{{\faLinkedin} \linkedIn}
}

\begin{document}

% \maketitle %% This is LaTeX's default title constructed from \title,\author,\date

\makecvtitle %% This is a custom command constructing the CV title from \cvname, \cvpersonalinfo

\section{ Education}
\cvsubsection{University of Cambridge | MEng in Computer Science}[Oct 2021 - June 2022]
Studying Courses in: Distributed Ledger Technologies, Overview of Natural Language Processing, Probabilistic Machine Learning, Advanced Topics in Machine Learning and Representation Learning on Graphs and Networks.
\cvsubsection{\textbf{University of Cambridge} | BA in Computer Science achieved Class I}[Sep 2018 - June 2021]
\begin{itemize}
    \item Awarded the \textbf{Bateman Scholarship} for excellent performance in Exam. 
    \item 3rd Year(\textit{Part II}): achieved \textbf{Class I}  \textbf{(}focused on Info Theory, Computer Vision, Machine Learning and Quantum Computing, with dissertation in Cross-Lingual Summarization--Investigating the effect of Translation on Summarization\textbf{)}
    % \textbf{Focusing on}: Advanced Algorithms, Bio-informatics, Information Theory, Principles of Communication, Digital Signal Processing, Computer Vision, Machine Learning and Bayesian Inference, Quantum Computing \& Deep Neural Networks.
    \item 2nd Year(\textit{Part 1}B): achieved Class I mark (remained unclassed due to COVID).
    % \item 1st Year(\textit{Part 1A}): achieved Class II.1 (Upper Second)
    
    % \item Previously Covered: Artificial Intelligence, Computer Networking, Concurrent and Distributed Systems, Further Java, Machine Learning \& Security
\end{itemize}

% \cvsubsection{Amity International School | CBSE-AISSE}[Mar 2016 - Mar 2018]
% \begin{itemize}
%     \item Math-99/100; Computer Science-99/100; Physics 96/100; Chemistry-96/100; English-93/100.
% \end{itemize}

%%%%%%%%%%%%%%%%%%%%%%%%%%%%%%%%%%%%%%%%%%%%%%%%%%%%%%%%%%%%%%%%%%%%

\section{ Experience}

\cvsubsection{On-Device Learning Intern | Department of Computer Science and Technology}[July 2021 - Sep 2021]
\begin{itemize}
    \item Investigated and Summarized possible reasons for latency in GPU, thus, expanding on the research already done. 
    \item Developed and Optimized matrix multiplication on Mobile GPUs leading to a 200\% increase in computation speed
    \item Utilized further tricks to make the GPU faster than the baseline (CPU).
    % \item Collaborated with researchers and PhD students. 
\end{itemize}

% \cvsubsection{Webmaster of Trinity Hall JCR | Student Committee Trinity Hall}[Feb 2020 - Feb 2021]
% \begin{itemize}
%     % \item Provided technical support to the JCR.
%     \item Revitalized the website to make access to information easier. (Traffic increased by 30\%)
%     \item Improved and Streamlined the balloting system. 
% \end{itemize}

\cvsubsection{Natural Language Processing Intern | Department of Computer Science and Technology}[June 2019 - Aug 2019]
\begin{itemize}
    \item Developed a customizable chat-bot using Diaglogflow which eliminated 80\% of the programming required.
    \item Upgraded the graphics kernel for the Virtual Environment making the experience smoother. 
    \item Headed a MineCraft Mod project in a Hackathon with Cambridge Assessment which won the Best Project prize.
    % \item developed skills around python (nltk and machine learning) and C# while working with Unity.
\end{itemize}

%%%%%%%%%%%%%%%%%%%%%%%%%%%%%%%%%%%%%%%%%%%%%%%%%%%%%%%%%%%%%%%%%%%%

\section{ Projects}

% \cvsubsection{MNN}[July 2021 - Sep 2021]
% \begin{itemize}
%     \item 
% \end{itemize}

\cvsubsection{Dissertation Project}[Oct 2020 - May 2021]
\begin{itemize}
    \item Conceptualized and Executed an Investigation of the effect of Translation quality on Summarization quality. 
    \item Built an understanding of the core architecture involved in the project.
    \item Evaluated the resultant translations and reported the findings in a report. Which achieved a first class mark.
\end{itemize}

\cvsubsection{pyopencl-ml}[Aug 2020]
\begin{itemize}
    % \item
    \item Investigated how to program a OpenCL Kernel \& how to build and train an Artificial Neural Network.
    \item Engineered a machine learning library in python compatible with all hardware accelerators using pyopencl.
    \item Improved GPU performance on Matrix Multiplication to be 10x faster than baseline-CPU (numpy).
    
\end{itemize}

\cvsubsection{manga-viewer}[May 2020]
\begin{itemize}
    % \item
    \item Studied web-scraping with Selenium and concurrent programming in python. 
    \item Implemented a web-scraper with Selenium and a GUI with Tkinter in python for scraping and displaying manga.
    % \item learnt about concurrent programming and working with futures
\end{itemize}

\cvsubsection{Automatic Evaluation of R code | 2nd Year Group Project}[Mar 2020]
\begin{itemize}
    % \item
    \item Headed a group of 6 to build a novel tool for automatic evaluation of R. Achieved 2nd highest score.
    \item Analyzed core-requirements of Cambridge Spark's EduKate.AI and Assigned roles to the team-members.
    \item Implemented the validation tool for automatically evaluate R code (check for semantic correctness).
    % \item Developed using rpy2 in Python a tool to validate R code i.e. check for syntactic errors.
    % \item Learnt valuable software development skills like making sure code is readable and different testing methods.
\end{itemize}

% \cvsubsection{Keep Talking and Nobody Explodes | GameGig Hackathon}[Dec 2019]
% \begin{itemize}
%     \item Developed a game in Lua with many mini games where the player has to accomplish certain goals to survive.
%     \item Learnt how to co-ordinate with multiple team members as a team lead and some Version Control practices.
%     % \item Built a game in Lua
%     % \item the team made multiple mini games which were then combined, and players have to have a minimum score in each game to win.
%     % \item Learnt a lot about working with multiple contributors on git
% \end{itemize}

% \cvsubsection{Mirror Task in Minecraft | Cambridge Assessment Hackathon}[Aug 2019]
% \begin{itemize}
%     \item Novice English speakers are put into two rooms in minecraft and the must communicate to make the rooms identical. 
%     % \item to only allow English through a language detector needed to be built.
%     \item Modded minecraft using Spigot in Java in order to extend the inputs to include some meta commands.
%     \item Developed a detector to only allow English test through.
% \end{itemize}

%%%%%%%%%%%%%%%%%%%%%%%%%%%%%%%%%%%%%%%%%%%%%%%%%%%%%%%%%%%%%%%%%%%%

\section{ Skills}

\cvsubsection{Languages}
\begin{itemize}
    \item \textbf{Python} with experience in scientific computing; concurrent programming \& ML (\textbf{\textit{Passed Linked-In assessment}}) 
    \item \textbf{Java} with experience in serialization, reflection \& class loaders; along with concurrency and synchronization
    \item \textbf{C \& C++} embedded programming; multi-threading \& multiprocessing. (\textbf{\textit{Passed LinkedIn Assessment}}: Top 5\%)
    \item Familiar with: C\# (with unity);{OpenCL}; Ocaml; HTML/CSS/JavaScript; Prolog; 
\end{itemize}
\hline
% \cvsubsection{Key skills}
\begin{itemize}
    \item \textbf{Software Development} using object-oriented design; testing; version control \& solving problems with logical thinking.
    \item \textbf{Native C++} development on Android using \textbf{Android Studios}; Contributing to the alibaba/MNN project: OpenCL Core.
    \item \textbf{Natural Language Processing}: Undertook undergraduate-dissertation project; further pursuing research in Master's.
    \item \textbf{Machine Learning} (\textbf{\textit{Passed Linked-In assessment}}) with Tensorflow/PyTorch: designed LSTMs for analyzing VIX to build a portfolio from the SP500; Achieved {\textit{Kaggle Certificates}} in Intermediate Machine Learning; worked on Transformers \& MobileNetv2
    
\end{itemize}
%\textbf{Other Interests}: Cricket, Formula 1, Chess, Rubik's Cube \& Volunteering.

% \cvsubsection{Amgen}[Thousand Oaks]
% [Scientist I][Sept 2018 to present]

% \section{Summary}
% \begin{itemize}
% \item Interdisciplinary scientist with skills and experience in immunology, genomics, and molecular biology 
% \item Led collaborative projects, resulting in 6 peer-reviewed publications, including 5 high impact first-authored publications, and 3 patents
% \item Deep understanding of genomic data analysis and visualization 
% \item Self-motivated, problem-solving and collaborative scientist with excellent communication skills
% \item Looking to contribute to use computational methods to push forward Gene Therapy projects towards the clinic
% \end{itemize}
 
% \section{Technical Skills}

% \begin{itemize}
% \item \textbf{Animal Handling:} Mouse handling, tissue harvest, IV/IP/IM/SC injections
% \item \textbf{Cell Biology:} Cell culture, Cell assays, Cell engineering, Cell fractionation, 
% \item \textbf{Microscopy/Imaging:} Confocal microscopy, Cell imaging and analysis 
% \item \textbf{Immunology:} Primary immune cell isolation and culture, Flow Cytometry (FACs) analysis, FACS sorting, 
% \item \textbf{Biochemistry:} ELISA, Western Blotting, Enzymatic assays
% \item \textbf{Molecular biology:} Cloning, PCR, RT/qPCR, Transfection, mini prep, AAV, Retroviral transduction, CRISPR 
% \item \textbf{Genomics:} RNA library construction for high throughput sequencing 
% \item \textbf{Computational:} Programming languages (R, Python and Shell script)
% \end{itemize}
 
% \section{Research Experience}

% %% Another custom command provide by scimisc-cv.sty.
% %% First two argumetns are typeset on the first line in bold; 3rd and 4th arguments are typset on second line in italics. 2nd, 3rd and 4th arguments are OPTIONAL


% \begin{itemize}
% \item Led 3 highly collaborative projects all focused on the validation of novel therapeutic vectors in animal disease models (neurodegenerative diseases)
% \item Managed a small team of 2 technical reports
% \item Responsible for designing experiments that drove the project forward towards IND submission
% \item Oversaw the PK/PD, and toxicology studies conducted by various CROs
% \item This project led to the submission of 3 publications and 1 patent
% \end{itemize}

% %% An example of leaving an argument empty
% \cvsubsection{Massachusetts General Hospital}[][Post doctoral Fellow][July 2014 to Sept 2018]

% \begin{itemize}
% \item Led 2 primary projects focused on the developing a library of small molecules targeting pathways involved in neurodegenerative diseases
% \item Developed high-throughput screening assays with novel functional readout (target validation assays)
% \item Used computational methods to develop novel small molecules that fit target profile
% \item These projects led to the submission of 2 publications and 2 patents
% \end{itemize}

 
% \section{Education}

% \begin{itemize}
% \item PhD, Computational/Molecular Biology, Harvard University, 2014  
% \item BS, Biology, University of Massachusetts, 2010
% \end{itemize}
 
% \section{Teaching and Mentoring Experience }
% \begin{itemize}
% \item 2014 - Mentored 2 undergraduates in their day-to-day lab activities
% \item 2012 - Graduate Teaching Assistant for Immunology
% \end{itemize}

% \section{Awards}
% \begin{itemize}
% \item Graduate Scholarship 
% \item F32: NIH Postdoctoral Training Grant
% \end{itemize}

% \section{Conference Presentations }

% Take the top 3-4
% \begin{itemize}
% \item Keystone Conference for Neurodegenerative Diseases:
% \end{itemize}

 
% \section{Publications}
% Take the top 5-6, bold your author position 


% \section{Other Skills}
% \begin{description}[widest=Langauges]
% \item[Software]	GraphPad Prism, Microsoft Word, Excel, and PowerPoint, ImageJ
% \item[Languages]	English: professional proficiency.  Mandarin: native.  German: conversational.
% \end{description}


\end{document}

